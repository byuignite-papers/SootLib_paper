%! suppress = MathOperatorEscape
%%
%% Copyright 2007, 2008, 2009 Elsevier Ltd
%% 
%% This file is part of the 'Elsarticle Bundle'.
%% ---------------------------------------------
%% 
%% It may be distributed under the conditions of the LaTeX Project Public
%% License, either version 1.2 of this license or (at your option) any
%% later version.  The latest version of this license is in
%%    http://www.latex-project.org/lppl.txt
%% and version 1.2 or later is part of all distributions of LaTeX
%% version 1999/12/01 or later.
%% 
%% The list of all files belonging to the 'Elsarticle Bundle' is
%% given in the file `manifest.txt'.
%% 

%% Template article for Elsevier's document class `elsarticle'
%% with numbered style bibliographic references
%% SP 2008/03/01

\documentclass[preprint,letterpaper]{elsarticle}

%% Use the option review to obtain double line spacing
%% \documentclass[authoryear,preprint,review,12pt]{elsarticle}

%-------- packages --------
\usepackage{amsmath}
\usepackage{amssymb}
%\usepackage{hyperref}
\usepackage{lineno}                 % line numbers
\usepackage[version=4]{mhchem}      % chem formatting
\usepackage{siunitx}                % units formatting
\usepackage{rotating}
%\usepackage{longtable}
%\usepackage{float}
%\usepackage{caption}
%\usepackage{xltabular}
\usepackage{tabularx}
\usepackage{fontawesome}
\usepackage{enumitem}
\setlist{itemsep=0pt}

%-------- package setup --------
\sisetup{exponent-mode = scientific}

%-------- front matter --------
\journal{SoftwareX}

\begin{document}

\begin{frontmatter}

%% Title, authors and addresses

%% use the tnoteref command within \title for footnotes;
%% use the tnotetext command for theassociated footnote;
%% use the fnref command within \author or \address for footnotes;
%% use the fntext command for theassociated footnote;
%% use the corref command within \author for corresponding author footnotes;
%% use the cortext command for theassociated footnote;
%% use the ead command for the email address,
%% and the form \ead[url] for the home page:
%% \title{Title\tnoteref{label1}}
%% \tnotetext[label1]{}
%% \author{Name\corref{cor1}\fnref{label2}}
%% \ead{email address}
%% \ead[url]{home page}
%% \fntext[label2]{}
%% \cortext[cor1]{}
%% \address{Address\fnref{label3}}
%% \fntext[label3]{}

\title{SootLib: a soot model library for combustion CFD}

%% use optional labels to link authors explicitly to addresses:
%% \author[label1,label2]{}
%% \address[label1]{}
%% \address[label2]{}

%\renewcommand{\thefootnote}{\fnsymbol{footnote}}
\author{Victoria B. Stephens}
\author{Josh Bedwell}
\author{David O. Lignell\corref{cor1}}

\cortext[cor1]{Corresponding author \ead{davidlignell@byu.edu}}

\address{Chemical Engineering Department, Brigham Young University, Provo, UT 84602, USA}

\begin{abstract}
sootlib abstract goes here
\end{abstract}

\begin{keyword}
soot \sep combustion \sep reacting flow simulation
\end{keyword}

\end{frontmatter}

\linenumbers

%-------- main text --------

%%%%%%%%%%%%%%%%%%%%%%%%%%%%%%%%%%%%%%%%%%%%%%%%%%%%%%%%%%%%%%%%%%%%%%%%%%%

\section{Introduction}
\label{s:intro}

Most commercial combustion processes that produce energy involve turbulent, non-premixed flames, which produce soot. Soot is responsible for a flame's luminosity, generates a large portion of a flame's radiative heat transfer to its surroundings, and contributes to many of the health, safety, and environmental hazards associated with air pollution from combustion systems~\cite{EPA_2009,EPA_2004}. In order to address soot's negative effects and optimize practical combustion processes, scientists and engineers seek a better understanding of soot's fundamental structure and behavior in combustion environments, often through modeling and simulation.
%Because combustion processes are so complex, modeling and simulation cannot be fully separated from studying fundamental chemical processes; instead, we must use incomplete knowledge and imperfect models to investigate both simultaneously. We often rely on experimental data for comparison, and as a result, errors must be distinguished by their source\textemdash experimental, theoretical, or computational\textemdash which further complicates modeling.
Turbulent combustion simulation is uniquely challenging because it involves tightly coupled equations of multicomponent mass transfer, convective and radiative heat transfer, turbulent fluid dynamics, multi-phase flow, and complex chemical kinetics. These span many orders of magnitude in both their length and time scales, and simulating soot in flames further expands the range of scales that must be considered, adding additional complexity and computational cost.

Direct simulation approaches can produce accurate simulation data by resolving the full range of length and time scales, but the computational cost can be prohibitively high, particularly for simulating practical combustion processes of interest to engineers, which includes most turbulent, non-premixed flame configurations~\cite{Pope_2000}.
%High computational cost in combustion simulations is often addressed by modeling various aspects of the configuration.
Computational models that quantify soot production in simulations can help us study its fundamental behavior and distinguish between various reaction mechanisms and transport models while also reducing the potentially high computational cost. Reactions involving soot are not purely chemical, but also involve particle aggregation, size distributions, and transport that may or may not affect molecular reactions and heat transfer within a flame. Accurate models and simulations that help us clarify the fundamental processes that control soot formation and transport in flames represent an important step forward in the study of combustion systems~\cite{Frenklach_2002b}.

With SootLib, we provide a convenient, easy-to-use access point for soot property and particle dynamics models that can be interfaced with various simulation approaches for combustion CFD. SootLib is a modular C++ library that can compute various source terms for the reacting Navier Stokes equations. Its modular design makes model parts interchangeable (within limits), allowing users to quickly and easily compare and contrast models. Several [TODO: put a number here] models are fully implemented and validated and use a common interface, allowing researchers convenient access to soot modeling tools suitable for various reacting CFD applications.

%%%%%%%%%%%%%%%%%%%%%%%%%%%%%%%%%%%%%%%%%%%%%%%%%%%%%%%%%%%%%%%%%%%%%%%%%%%

\section{Model Descriptions}
\label{s:models}

Soot models can generally be broken down into two interrelated parts: soot chemistry and particle size distribution (PSD). Soot chemistry models address the chemical reactions involved in soot behavior, including particle inception, growth, and destruction, while PSD models describe the size distribution of soot particles and the rates at which the distribution changes. Given a thermodynamic state and a set of chemical mechanisms, the PSD model predicts the amounts and relative locations of variously sized soot particles based on their current distribution, the desired chemistry, and the thermodynamic state of the surrounding gas. Using a particular set of soot chemistry models generally does not necessitate using a particular PSD model, and vice versa. SootLib takes advantage of this distinction by allowing users to specify individual models rather than predetermined model sets, giving users more flexibility and model developers more control in simulations.

%--------------------------------------------------------------------------
\subsection{Chemistry}
\label{ss:chemistry}

In a modeling and simulation context, soot is defined as a relatively large collection of carbon atoms that may also contain lesser amounts of hydrogen and other elements. Soot particles are usually assumed to be spherical, which facilitates easy calculation of particle diameter, volume, and surface area; this is a reasonable assumption for nascent and relatively small soot particles, but may not be adequate to describe large soot agglomerates, which tend to exhibit fractal structures~\cite{Jullien_1987}[CHECK CITATION].

Soot chemistry is typically divided into four primary types: nucleation reactions describe how soot particles form from gaseous precursors;
%and may or may not include the condensation of large polycyclic aromatic hydrocarbons (PAH);
surface growth refers to particle growth by the chemical addition of gaseous species to existing soot particles; oxidation reactions describe soot particle size reduction by reactions with oxygen-based gaseous species; and coagulation defines the rate at which soot particles combine to form new particles.
%(and may or may not include the aggregation of particles into large fractal soot structures).
Soot may also undergo additional processes such as growth by condensation of large polycyclic aromatic hydrocarbons (PAH) or agglomeration of particles into large fractal aggregates, but such reactions may not be significant in all scenarios and configurations.
A complete soot chemistry model typically provides a mechanism for each of the four major steps, each of which occurs independently but depends on the composition and thermodynamic properties of the surrounding gas. Some models provide all four mechanisms, while later research tends to expand on earlier mechanisms or focus on one type. SootLib collects a range of models from the literature and implements them with a uniform interface, offering model developers a flexible testing method and giving research simulations a consistent framework for parametric simulations. Table~\ref{t:chem_models} summarizes the soot chemistry models implemented in SootLib.
%Table~\ref{t:variables} defines the variables used in the presented models.

\begin{sidewaystable}
    \caption{Summary of soot chemistry models implemented in SootLib. In mechanisms, \ce{C(s)} represents a soot particle, i.e. "solid" carbon, and \ce{C(s)^.} indicates a radical site on a soot particle, typically caused by hydrogen abstraction. Variable definitions for rate expressions can be found in~\ref{s:nomenclature}. Rate expressions that involve multiple expressions or otherwise do not fit well in this table can be found in the appendices as noted.}
    \label{t:chem_models}
    \centering
    \resizebox{\textwidth}{!}{
        \begin{tabular}{l l l l l}
            \hline
            Chemistry type  & Model                 & Model ID     & Mechanism & Rate expression \\
            \hline \hline
            Nucleation      & Leung \& Lindstedt~\cite{Leung_1991}  & \texttt{LL}    & \ce{C2H2 -> 2C(s) + H2} & $R_{nuc} = \num{0.1e5} e^{-21100/T} \ce{[C2H2]}$\\
                            & Lindstedt 2005~\cite{Lindstedt_2005}  & \texttt{LIN}   & \ce{C2H2 -> 2C(s) + H2} & $R_{nuc} = \num{0.63e4} e^{-21100/T} \ce{[C2H2]}$\\
                            & PAH nucleation~\cite{Blanquart_2009}  & \texttt{PAH}   & dimerization of gaseous PAH  &\\
            \hline
            Surface growth  & Leung \& Lindstedt~\cite{Leung_1991}  & \texttt{LL}    & \ce{C2H2 + nC(s) ->} (n+2)\ce{C(s) + H2} & $R_{grw} = \num{0.6e4} e^{-21100/T} f(S) \ce{[C2H2]}$\\
                & Lindstedt 1994~\cite{Lindstedt_1994}  & \texttt{LIN}   & \ce{C2H2 + nC(s) ->} (n+2)\ce{C(s) + H2} & $R_{grw} = \num{0.1e-11} e^{-12100/T} \ce{[C2H2]} 2M_0 MW_C$ \\
                & HACA~\cite{Appel_2000,Frenklach_1994} & \texttt{HACA}  & \ce{C(s)-H + H <=> C(s)^. + H2}   & $R_{grw,f}=\num{4.2e13} e^{-13/RT} \ce{[H]}$ \\
                &                                       &                &                                   & $R_{grw,r}=\num{3.9e12} e^{-11/RT} \ce{[H2]}$ \\
                &                                       &                & \ce{C(s)^. + H -> C(s)-H}         & $R_{grw}=\num{2.0e13} \ce{[H]}$ \\
                &                                       &                & \ce{C(s)^. + C2H2 -> C(s)-H + H}  & $R_{grw}=\num{8.0e7} T^{1.56} e^{-3.8/RT} \ce{[C2H2]}$\\
                &                                       &                & \ce{C(s)-H + OH <=> C(s)^. + H2O} & $R_{grw,f}=\num{1e10} T^{0.734} e^{-1.43/RT} \ce{[OH]}$ \\
                &                                       &                &                                   & $R_{grw,r}=\num{3.68e8} T^{1.139} e^{-17.1/RT} \ce{[H2O]}$ \\

            \hline
            Oxidation       & Leung \& Lindstedt~\cite{Leung_1991}   & \texttt{LL}   &  \ce{C(s) + 1/2O2 -> CO} & $R_{oxi,\ce{O2}} = \num{0.1e5} T^{1/2} e^{-19680/T} f(S) \ce{[O2]}$\\
                            & HACA~\cite{Appel_2000,Frenklach_1994} & \texttt{HACA}  & \ce{C(s)^. + O2 -> 2CO + products} & $R_{oxi,\ce{O2}}=\num{2.2e12} e^{-7.5/RT} \ce{[O2]}$\\
                            &                                       &                & \ce{C(s)-H + OH -> CO + products} & $R_{oxi,\ce{OH}}=0.13*1290 P_{\ce{OH}} T^{-1/2} $\\
                            & Lee~\cite{Lee_1962} +
                              Neoh~\cite{Neoh_1980,Neoh_1981}       & \texttt{LEE\textunderscore NEOH} & \ce{C + 1/2O2 -> CO} & $R_{oxi,\ce{O2}} = \num{1.085e4} P_{\ce{O2}} T^{-1/2} e^{-19778.24/T}$\\
                            &                                       &                & \ce{C + OH -> CO + H} & $R_{oxi,\ce{OH}}=0.13*1290 P_{\ce{OH}} T^{-1/2}$ \\
                            & NSC~\cite{Nagle_1962} +
                              Neoh~\cite{Neoh_1980,Neoh_1981}       & \texttt{NSC\textunderscore NEOH} & \ce{C + 1/2O2 -> CO} & see~\ref{a:NSC}\\
                            &                                       &                & \ce{C + OH -> CO + H} & $R_{oxi,\ce{OH}}=0.13*1290 P_{\ce{OH}} T^{-1/2}$\\
            \hline
            Coagulation     & Leung \& Lindstedt~\cite{Leung_1991}  & \texttt{LL}    & \ce{nC(s) -> C_n(s)} & $R_{coa} = -2C_a d_p^{1/2} \left( \frac{6k_B T}{\rho_s}\right) (\rho N)^2$\\
                            & Fuchs~\cite{Fuchs_1964,Seinfeld_2016} & \texttt{FUCHS} & \ce{nC(s) -> C_n(s)} & see~\ref{a:FUCHS} \\
                            & Frenklach~\cite{Frenklach_2002b}       & \texttt{FRENK} & \ce{nC(s) -> C_n(s)} & see~\ref{a:FRENK} \\
            \hline
        \end{tabular}
    }
\end{sidewaystable}

%--------------------------------------------------------------------------
\subsubsection{Nucleation}
\label{sss:nuc}

Particle nucleation refers to the mechanisms by which the smallest possible soot particles are formed. The most common definition used in soot modeling literature defines a soot particle as having at least 100 carbon atoms~\cite{Leung_1991}; this is the value used by SootLib for kinetic-style nucleation mechanisms. The minimum soot particle size may also be defined by the nucleation mechanism itself, such as in SootLib's PAH nucleation mechanism, which defines the smallest soot particles as the result of a collision between two PAH dimers. The simplest nucleation models link nucleation rate to the concentration of acetylene (\ce{C2H2}) in the gas mixture, while more complex models may account for any number of gaseous hydrocarbon species.

SootLib's simplest chemistry model is the simplified kinetic mechanism presented by Leung and Lindstedt (\texttt{LL}), which consists of four Arrhenius-style rate expressions: one each for soot nucleation, surface growth, oxidation, and coagulation~\cite{Leung_1991}. The \texttt{LL} nucleation step relies on the concentration of gaseous acetylene and an empirical rate constant to calculate the nucleation rate. Lindstedt later proposed an alteration to the \texttt{LL} nucleation step's pre-exponential factor to increase accuracy without changing the expression's form (\texttt{LIN})~\cite{Lindstedt_2005}. Both of these are empirical models known for simplicity and speed rather than accuracy, but can be appropriate under certain conditions.

Experimental literature indicates that soot produced by gaseous fuels tends to nucleate via PAH collisions rather than directly from acetylene precursors, though the exact mechanisms are unknown and depend on the fuel, its surrounding environment, and various other factors [CITATION]. SootLib implements the PAH nucleation model presented by Blanquart and Pitsch (\texttt{PAH}), which represents a more physics-based approach to soot particle nucleation in which incipient soot particles are created by the collision of two PAH dimers, which are themselves created by the collision of two PAH molecules~\cite{Blanquart_2009c}. This model also accounts for condensation of PAH dimers onto existing particles, which represents an additional surface growth mechanism.

%--------------------------------------------------------------------------
\subsubsection{Surface growth}
\label{sss:grw}

Surface growth refers to the addition of carbon atoms to an existing soot particle from gaseous hydrocarbons. Simple surface growth models typically rely on acetylene (\ce{C2H2}) as the primary source of gaseous carbon, though other gaseous hydrocarbons may also contribute to particle surface growth to varying degrees. Additionally, surface growth models tend to include some dependence on the soot particle surface area since the availability of sites for addition of carbon atoms to existing particles tends to be a limiting factor in rate calculations.

The Leung and Lindstedt mechanism for soot surface growth (\texttt{LL}) depends on both the gaseous concentration of acetylene and the available particle surface area upon which surface reactions can occur. Specifically, it is assumed that the number of local active sites available for surface growth and oxidation reactions is proportional to the square root of the total particle surface area in the flame. If the surface area is given by $S=\pi d_p^2 \rho_s N$ and the soot particle diameter by $d_p=(6Y_{s}/\pi \rho_{s} N)^{1/3}$, this surface area dependence becomes
\begin{equation}
    \label{e:surface_area}
    f(S) = \sqrt{\pi \left( \frac{6MW_C}{\pi \rho_{s}} \right) ^{2/3}} \left[ \frac{\rho Y_{s}}{MW_C} \right]^{1/3} [\rho N]^{1/6},
\end{equation}
where $MW_C$ is the molar mass of carbon, $\rho$ is the gas density, $\rho_{s}$ is the solid density of soot, $Y_{s}$ is the mass fraction of soot, and $N$ is the soot particle number density~\cite{Leung_1991}. Lindstedt (\texttt{LIN}) also proposed a modified version of the \texttt{LL} surface growth rate that eliminates its dependence on particle surface area in favor of dependence on the particle number density~\cite{Lindstedt_1994}.

The hydrogen-abstraction acetylene-addition (\texttt{HACA}) model presents a more detailed mechanism for both surface growth and oxidation. It breaks down the global dependence of surface growth on acetylene into elementary reaction steps, each with its own Arrhenius-style rate expression, decreasing the empiricism of the model and potentially increasing its accuracy compared to simpler mechanisms~\cite{Appel_2000}. In doing so, it also introduces a dependence on other gaseous species, including \ce{H2}, \ce{H2O}, and \ce{O2}. The \texttt{HACA} model represents a good balance between accuracy and computational cost, making it a popular choice for soot modeling in simulations.

%--------------------------------------------------------------------------
\subsubsection{Oxidation}
\label{sss:oxi}

Soot oxidation refers to the process by which soot particles decrease in size due to reactions with gaseous species. Similar to surface growth, oxidation mechanisms may depend on the available surface area of oxidizing soot particles, which may or may not be a limiting factor depending on the thermodynamic state, composition of the gas mixture, and the amount of soot present. Early soot oxidation models tend to rely on \ce{O2} as the principal oxidant, though experimental studies show that \ce{OH} and sometimes \ce{O} can also contribute significantly to soot oxidation rates.

The Leung and Lindstedt ({\texttt{LL}}) oxidation model depends only on the concentration of gaseous \ce{O2} and an empirical rate constant. The \texttt{LL} oxidation expression is itself based on an earlier model presented by Lee et al.~\cite{Lee_1962}, which models the oxidation of soot particles by \ce{O2} with a global rate expression. Nagle and Strickland-Constable~\cite{Nagle_1962} also presented a commonly used model for soot oxidation by \ce{O2} in which the rate of oxidation depends on a nonlinear combination of Arrhenius-style rate constants applied to the partial pressure of \ce{O2} (see~\ref{a:NSC}). These early soot oxidation models, however, do not take into account the influence of oxidation by \ce{OH}, which can be significant for certain flame configurations and fuels~\cite{Neoh_1980}. To account for this, SootLib adds expressions for \ce{OH} oxidation presented by Neoh et al.~\cite{Neoh_1981} to the oxidation expressions presented by Lee et al. and Nagle and Strickland-Constable, resulting in the \texttt{LEE\textunderscore NEOH} and \texttt{NSC\textunderscore NEOH} options for soot oxidation in SootLib. Leung and Lindstedt explicitly acknowledge the lack of oxidation by \ce{OH} in their model, but consider their one-step oxidation mechanism sufficient for their purposes; in light of their reasoning and the widespread use of their model as published, SootLib does not alter or add to the existing Leung and Lindstedt oxidation step. SootLib also includes the \texttt{HACA} mechanism for soot oxidation, as discussed above~\cite{Appel_2000}.

%--------------------------------------------------------------------------
\subsubsection{Coagulation}
\label{sss:coa}

Coagulation refers to the process by which soot particles increase in size due to collisions with other soot particles. It is closely related to the soot particle size distribution (PSD), which describes the population of soot particles in terms of size, mass, and number density. While most coagulation processes are not technically chemical reactions, coagulation models are typically structured so as to be analagous to the strictly chemical steps and are usually categorized as a type of soot chemistry. At present, SootLib does not include any coagulation models that account for the agglomeration of soot particles into large fractal aggregates.

SootLib includes three coagulation mechanisms, all of which interact strongly with the particle size distribution (PSD) model, discussed in Section~\ref{ss:PSD_dynamics}. The simplest coagulation model is that presented as part of the Leung and Lindstedt model (\texttt{LL}), which describes particle coagulation with a normal square dependence using an empirical agglomeration rate coefficient~\cite{Leung_1991}.

Physics-based coagulation models represent the coagulation rate of two particles with the expression $R_{coa,12}=K_{12}N_1N_2$, where $N_1$ and $N_2$ are the number density of particles 1 and 2, respectively. $K_{12}$ is coagulation coefficient given by $K_{12}=2\pi (D_{p1}+D_{p2})(D_1+D_2)$, where $D_{p1}$ and $D_{p2}$ are the diameters and $D_1$ and $D_2$ are the Brownian diffusivity of particles 1 and 2, respectively~\cite{Seinfeld_2016}. Particle diffusivity, in turn, depends on the surrounding conditions. In the continuum regime, when the mean free path $\lambda_p$ of a diffusing particle is comparable to or less than its radius ($Kn\ll1$), its diffusion can be described by the Einstein-Stokes relation; in the free-molecular regime, when the mean free path $\lambda_p$ of the particle is much larger than its radius ($Kn\gg1$), diffusion can be described with kinetic theory. Depending on the external conditions, soot particle coagulation can occur at either limit or in between, known as the transition regime.

Fuchs (\texttt{FUCHS}) proposed a generalized coagulation coefficient of the form $K_{12}=2\pi (D_{p1}+D_{p2})(D_1+D_2)\beta$, where $\beta$ is a correction factor that accounts for the kinetic effect of the particle regime (see~\ref{a:FUCHS})~\cite{Fuchs_1964, Seinfeld_2016}. In the continuum and free-molecular limits, the coagulation coefficient simplifies to the value given by the Einstein-Stokes and kinetic theory expressions, respectively. Frenklach (\texttt{FRENK}) uses a harmonic mean of the continuum and free-molecular values to represent the transition regime, reporting that the approach reproduces results from the Fuchs form of the coagulation coefficient within about 20\% accuracy~\cite{Frenklach_2002b,Kazakov_1998}.

%--------------------------------------------------------------------------
\subsection{Particle size distribution and dynamics}
\label{ss:PSD_dynamics}

In addition to the chemical reactions that influence their creation and destruction, soot particles also undergo non-chemical changes, typically in the form of coagulation, aggregation, and their inverse processes. As a result, it becomes computationally complex to treat each possible combination of carbon atoms as its own chemical species. Instead, we describe the collection of soot particles as a whole with a particle size distribution (PSD). There are two common approaches to the soot PSD: direct/sectional models and moment methods. Direct methods account for every possible particle size, and sectional methods divide the domain of possible soot particle sizes into discrete ranges based on size or mass; in both cases, the number density of soot particles of each increases or decreases according to defined soot chemistry and mechanisms. The more divisions that are defined in a sectional method, the closer the approximation gets to describing the true distribution (as a direct method would), but the higher the computational cost becomes.

Moment methods, on the other hand, describe the PSD using the statistical moments of the distribution, where the $k^{th}$ mass moment of the discrete soot PSD is defined by
\begin{equation}
    M_k = \sum_{i=1}^{\infty} m_i^k N_i,
\end{equation}
where $m$ is the mass and $N$ is the number density of a given soot particle of size $i$. In most practical applications, only a small number of moments (2--8) is required to describe the full PSD. As a result, moment methods are generally much more numerically efficient than sectional methods. Moment transport equations take the form
\begin{equation}
    \label{e:momTransEq}
    \frac{\partial M_k}{\partial t} + \frac{\partial v_k M_k}{\partial x_k} = -\frac{\partial M_k V_k}{\partial x_k} + S_k,
\end{equation}
where the velocity $v_k$ and the diffusion velocity $V_k$ are assumed to be independent of particle size (which is a reasonable assumption for small soot particles transported mainly by convection and thermophoresis). $S_k$ represents the moment source term, which is comprised of the sum of individual moment source terms for the soot chemistry steps, usually one each for nucleation, surface growth, oxidation, and coagulation. While moment equations are exact, their source terms are unclosed---typically by requiring fractional moment values---and further knowledge of the size distribution is required to calculate them. Moment methods differ primarily in their approach to the source term closure problem.

SootLib includes four moment models for calculating the source terms of the moment transport equations, summarized in~\ref{t:psd_models}. Two of the models---\texttt{MONO} and \texttt{LOGN}---are assumed-shape size distribution models in which we assume a particular shape for the soot PSD in order to close the source terms. The two remaining models---\texttt{QMOM} and \texttt{MOMIC}---do not assume a shape for the PSD; instead, they obtain values for the unclosed fractional moments via numerical quadrature (\texttt{QMOM}) or interpolation (\texttt{MOMIC}). At the time of this writing, SootLib does not include either direct or sectional methods, but its structure is designed to be able to accommodate their addition in future releases.

\begin{table}
    \caption{Summary of soot particle size distribution models implemented in SootLib.}
    \label{t:psd_models}
    \centering
    \resizebox{\textwidth}{!}{
        \begin{tabular}{l l l}
            \hline
            PSD model                                    & Model ID & \# Moments   \\
            \hline
            Assumed monodisperse~\cite{Lignell_2008b}     & \texttt{MONO}  & 2      \\
            Assumed lognormal~\cite{Lignell_2008b}        & \texttt{LOGN}  & 3  \\
            Quadrature method of moments~\cite{McGraw_1997,Marchisio_2013} & \texttt{QMOM}  & 2, 4, 6  \\
            Method of moments with interpolative closure~\cite{Frenklach_2002b} & \texttt{MOMIC} & 2\textendash8  \\
            \hline
        \end{tabular}
    }
\end{table}

\subsubsection{Monodisperse distribution (\texttt{MONO})}
\label{sss:mono}

SootLib's simplest PSD closure model is the assumed monodisperse size distribution (\texttt{MONO}), which requires only the first two moments of the soot PSD. These two moments, $M_0$ and $M_1$, correspond to the overall particle number density and mass density of soot particles, respectively, and can be used to define an average particle diameter
\begin{equation}
    d = \left( \frac{6M_1}{\pi \rho_s M_0} \right)^{1/3},
\end{equation}
which allows reaction rates to depend on quantities related to particle size, including particle surface area. Fractional moments are computed via logarithmic interpolation between $M_0$ and $M_1$, closing the moment source terms with relatively simple analytic expressions~\cite{Lignell_2008b}.

Due to its simplicity and ease of use, the \texttt{MONO} model is a popular starting point for soot PSD modeling; it is implemented in SootLib for that reason. Its accuracy may be sufficient for simple flame configurations that produce relatively small soot particles in relatively small quantities. Because this model assumes a uniform shape for the soot PSD, however, it tends to lose accuracy as the distribution of soot particles becomes more varied and complex.

\subsubsection{Lognormal distribution (\texttt{LOGN})}
\label{sss:logn}

Experimental studies show that the soot PSD early in a flame's lifetime may be reasonably approximated by a lognormal distribution [CITATION].
The assumed lognormal distribution (\texttt{LOGN}) model builds on the simplicity of the \texttt{MONO} model by adding a third mass moment, $M_2$, and assuming that the soot PSD can be described by a lognormal distribution rather than a monodisperse distribution~\cite{Lignell_2008b}. Whole-order moments can be calculated based on predefined parameters for the lognormal distribution in combination with $M_0$, and fractional moments can be calculated from the whole-order moments as
\begin{equation}
    M_k = M_0^{1-\frac{3}{2}k+\frac{1}{2}k^2} M_1^{2k-k^2} M_2^{\frac{1}{2}k^2-\frac{1}{2}k}.
\end{equation}

As with the \texttt{MONO} model, the \texttt{LOGN} model is straightforward and computationally efficient while also increasing the potential for accuracy by assuming a more physically viable shape for the soot PSD. Unfortunately, assuming the distribution's shape still causes the model to lose accuracy as the PSD increases in complexity, though it can be appropriate for cases involving relatively small soot particles that do conform to a roughly lognormal PSD.

\subsubsection{Quadrature method of moments (\texttt{QMOM})}
\label{sss:qmom}

Higher accuracy can be obtained by avoiding assuming the shape of the soot PSD, but this requires an alternate approach to closing the moment source terms. The quadrature method of moments (QMOM) does this by applying numerical quadrature to the unknown PSD~\cite{McGraw_1997}. SootLib applies the Wheeler algorithm for moment inversion to calculate the weights and abscissas of the unknown soot PSD~\cite{Marchisio_2013,Wheeler_1974}. For a set of $N$ whole-order moments, resulting in $N/2$ weights $w$ and abscissas $x$ from the inversion algorithm, fractional moments can be calculated with
\begin{equation}
    M_k = \sum_{i=1}^{N/2} w_i x_i^k.
\end{equation}

Avoiding specifying a shape for the soot PSD significantly increases the accuracy of the model, particularly when the distribution increases in complexity, and while this does increase the computational resources required, efficient inversion algorithms and careful implementation can reduce it to acceptable levels. Due to the nature of numerical quadrature, \texttt{QMOM} is limited to even-numbered moment sets ($N=2,4,6\ldots$).

\subsubsection{Method of moments with interpolative closure (\texttt{MOMIC})}
\label{sss:momic}

Like \texttt{QMOM}, the method of moments with interpolative closure (MOMIC) also avoids specifying the shape of the soot PSD, but it closes the moment source terms by interpolating between whole order moments to calculate fractional moments rather than applying any numerical quadrature. This potentially increases its numerical efficiency over quadrature methods while retaining similar levels of accuracy. SootLib implements MOMIC as described by Frenklach~\cite{Frenklach_2002b,Frenklach_1987}, which uses a Lagrange interpolation between logarithms of the  whole-order reduced moments, $\mu_k = M_k/M_0$.

%\subsubsection{Sectional model (SECT)}
%\label{sss:sect}


%%%%%%%%%%%%%%%%%%%%%%%%%%%%%%%%%%%%%%%%%%%%%%%%%%%%%%%%%%%%%%%%%%%%%%%%%%%

\section{Software description}
\label{s:architecture}

SootLib is an object-oriented C++ library intended for use in combustion CFD codes of various types. Upon download, the SootLib package contains five directories:
\begin{itemize}
    \item[\faFolderO] \texttt{src} contains the SootLib source code;
    \item[\faFolderO] \texttt{ext} contains externally sourced functions used by SootLib;
    \item[\faFolderO] \texttt{examples} contains illustrative example codes using SootLib;
    \item[\faFolderO] \texttt{tests} contains SootLib's optional testing suite, driven by Catch2; and
    \item[\faFolderO] \texttt{docs} contains the files optionally used to generate code documentation with Doxygen.
\end{itemize}

SootLib installation is automated by CMake. To build and install the library with the default settings, the user must create and navigate into the \texttt{build} directory and execute the following commands:
\begin{enumerate}
    \item \texttt{cmake ..}
    \item \texttt{make}
    \item \texttt{make install}
\end{enumerate}
Project options can be changed by editing the top-level \texttt{CMakeLists.txt} file or, after running the CMake configuration step at least once, by editing the \texttt{CMakeCache.txt} file located in the \texttt{build} directory. Refer to the package documentation for the full list of CMake options.

Successful installation will generate the following additional directories and files:
\begin{itemize}
    \item[\faFolderO] \texttt{include}
    \begin{itemize}
        \item[\faFolderO] \texttt{sootlib}
        \begin{itemize}
            \item[\faFileTextO] \texttt{constants.h}
            \item[\faFileTextO] \texttt{sootModel.h}
            \item[\faFileTextO] \texttt{state.h}
        \end{itemize}
    \end{itemize}
    \item[\faFolderO] \texttt{lib}
    \begin{itemize}
        \item[\faFolderO] \texttt{cmake}
        \begin{itemize}
            \item[\faFolderO] \texttt{sootlib}
            \begin{itemize}
                \item[\faFileCodeO] \texttt{sootlib.cmake}
            \end{itemize}
        \end{itemize}
        \item[\faFileCodeO] \texttt{libsootModel.a}
    \end{itemize}
\end{itemize}
To use SootLib in C++ code, include the header files located in the \texttt{include} directory and link to the \texttt{libsootModel.a} library file. SootLib can also be used as part of larger CMake projects via \texttt{sootlib.cmake} or CMake's FetchContent module.

The SootLib library consists primarily of two object classes through which the user interacts with the library---\texttt{state} and \texttt{sootModel}---both of which are contained within the \texttt{soot} namespace. The \texttt{state} object holds user-specified details about the current thermodynamic state in which the soot chemistry occurs, including variables such as temperature, pressure, and gas species mass fractions. The \texttt{sootModel} object contains information about the selected models and mechanisms and performs the calculations that generate moment source terms. In the context of a traditional CFD simulation, the \texttt{state} object would be updated via the \texttt{setState} function at each individual time step and/or grid point, while the \texttt{sootModel} parameters only need to be specified once when the object is created, and then its \texttt{calcSourceTerms} function invoked at each step following the \texttt{setState} update. The resulting moment source terms, which are calculated assuming that the moment transport equations adhere the form of Equation~\ref{e:momTransEq}, and gas species source terms can be accessed via the \texttt{sootModel} object. For more details on using the SootLib library, refer to the package documentation and examples.

SootLib's major functionalities can be summarized as follows:
\begin{itemize}
    \item Calculates moment source terms (Eq.~\ref{e:momTransEq}) for the soot PSD given thermodynamic state details and chemistry mechanisms chosen by the user; and
    \item Calculates mass fraction source terms for gaseous chemical species affected by soot chemistry, including PAH where appropriate.
\end{itemize}
Auxiliary functionalities and features include the following:
\begin{itemize}
    \item \texttt{str2-SootLibType} functions to facilitate conversion of string values to SootLib's internal enum types for gas species, PAH species, and mechanism specifiers;
    \item Custom \texttt{dimerStruct} structure for access to additional details related to PAH nucleation and condensation calculations; and
    \item Object-oriented, modular mechanism structure designed to facilitate extension to additional mechanisms and model types.
\end{itemize}

%%%%%%%%%%%%%%%%%%%%%%%%%%%%%%%%%%%%%%%%%%%%%%%%%%%%%%%%%%%%%%%%%%%%%%%%%%%

\section{Validation and Examples}
\label{s:examples}

Self preserving size distribution example

Use simple example code to compare models?

Validate against literature plots where possible

%Provide at least one illustrative example to demonstrate the major functions.

%%%%%%%%%%%%%%%%%%%%%%%%%%%%%%%%%%%%%%%%%%%%%%%%%%%%%%%%%%%%%%%%%%%%%%%%%%%

\section{Discussion}
\label{s:discussion}

%\textbf{This is the main section of the article and the reviewers weight the description here appropriately}
%
%Indicate in what way new research questions can be pursued as a result of the software (if any).
%
%Indicate in what way, and to what extent, the pursuit of existing research questions is improved (if so).
%
%Indicate in what way the software has changed the daily practice of its users (if so).
%
%Indicate how widespread the use of the software is within and outside the intended user group.
%
%Indicate in what way the software is used in commercial settings and/or how it led to the creation of spin-off companies (if so).
Global kinetic mechanisms, typically represented by Arrhenius-style rate expressions, are popular choices in combustion simulations involving soot, particularly in cases that require soot modeling but cannot afford the relatively high computational cost of detailed soot models or gas mechanisms. Because global models represent each type of soot chemistry with simplified and often empirical rate expressions, they do not capture all of the fundamental mechanisms of various soot phenomena, but they have the advantages of being relatively simple, easy to implement, and computationally inexpensive. In other words, global models tend to sacrifice accuracy in favor of high speed and low computational cost. Additionally, commonly used global models can serve as convenient points of reference when developing or testing modified or more complex soot models.

Physics-based models and mechanisms attempt to increased the accuracy of more empirical models by using multiple elementary reaction steps to represent actual soot behavior rather than relying on the empiricism inherent in global reaction models. By accounting for fundamental phenomena, physics-based models may result in increased accuracy at the cost of computational speed and efficiency. The two-step oxidation models presented above are simple examples of how using multiple reaction steps can increase accuracy; by accounting for oxidation by both \ce{O2} and \ce{OH}, such models can use validated rate data for elementary reaction steps, potentially resulting in higher accuracy under certain conditions than a model that only considers one oxidation species or combines both dependencies into one empirical rate expression.

\subsection{Model combinations and limitations}
\label{ss:limitations}

SootLib is designed to be somewhat modular in that the various soot chemistry mechanisms can be substituted and exchanged at will, providing users with increased flexibility. It must be noted, however, that not all model combinations will produce physically meaningful results, either due to a model's design or its author's intent. The following notes indicate possible points of friction and limitations that SootLib users must be aware of when choosing model combinations.

When using SootLib for combustion CFD, the gas-phase chemistry mechanism must include the gas species required by the chosen soot chemistry mechanisms. For instance, a one-step global ethylene mechanism for gas chemistry will generate zero-value source terms for the soot nucleation mechanisms because it does not include the acetylene or gaseous PAH required by SootLib's nucleation mechanisms. SootLib does not explicitly warn users when such situations occur, as there may be cases in which a user finds it advantageous or instructive to exclude or explicitly set a gas species mass fraction. Therefore, users must take care to choose soot chemistry mechanisms that are appropriate to their use case. One specific model to be aware of in this respect is the \texttt{PAH} nucleation mechanism, which uses the concentrations of a small subset of gaseous PAH molecules. If none of the relevant PAH species is present in the gas mechanism, no soot nucleation will occur. Refer to the SootLib documentation for more details.

The following points apply to individual mechanisms:
\begin{itemize}
    \item The \texttt{LL} model is included with SootLib as a point of reference because it is one of the most common global soot models used in combustion simulations. This model was developed using observations of laboratory-scale ethylene jet flames, and its accuracy and applicability are generally limited to conditions similar to those under which it was developed.
    \item The \texttt{LL} mechanisms were designed to be used together, but can still provide physically meaningful results when separated.
    \item The \texttt{HACA} surface growth and oxidation mechanisms were not designed to be used separately, but may still perform adequately. Use caution when separating them.
    \item The \texttt{PAH} nucleation mechanism also includes a PAH condensation mechanism. This is by design in order to more closely adhere to the model as presented by its authors~\cite{Blanquart_2009c}. At present, the two cannot be separated, though this may change in future releases.
\end{itemize}


%%%%%%%%%%%%%%%%%%%%%%%%%%%%%%%%%%%%%%%%%%%%%%%%%%%%%%%%%%%%%%%%%%%%%%%%%%%

\section{Conclusions}
\label{s:conclusions}

Set out the conclusion of this original software publication.

%%%%%%%%%%%%%%%%%%%%%%%%%%%%%%%%%%%%%%%%%%%%%%%%%%%%%%%%%%%%%%%%%%%%%%%%%%%

\section{Conflict of Interest}
%Please select the appropriate text:

%Potential conflict of interest exists:
%We wish to draw the attention of the Editor to the following facts, which may be considered as potential conflicts of interest, and to significant financial contributions to this work. The nature of potential conflict of interest is described below: [Describe conflict of interest]

%No conflict of interest exists:
The authors declare that they have no known competing financial interests or personal relationships that could have appeared to influence the work reported in this paper.

%%%%%%%%%%%%%%%%%%%%%%%%%%%%%%%%%%%%%%%%%%%%%%%%%%%%%%%%%%%%%%%%%%%%%%%%%%%

\section*{Acknowledgements}

%The authors extend special thanks to Hadi Bordbar for assistance with the WSGG model and to Vladimir Solovjov and Brent Webb for their insights and assistance with the RCSLW model.
This research did not receive any specific grant from funding agencies in the public, commercial, or not-for-profit sectors.

%%%%%%%%%%%%%%%%%%%%%%%%%%%%%%%%%%%%%%%%%%%%%%%%%%%%%%%%%%%%%%%%%%%%%%%%%%%

\section*{Nomenclature}
\label{s:nomenclature}

\noindent
{\small
\begin{tabularx}{\textwidth}{l >{\raggedright\arraybackslash}X l}
    \hline
    Variable        & Definition                               & Value \\
    \hline \hline
    $C_a$           & Agglomeration rate constant               & 9.0  \\
    $C_i$           & Cunningham slip correction factor         & \\
    $C_{min}$       & Minimum number of carbon atoms in an incipient soot particle & 100 \\
    $D_i$           & Brownian diffusivity of particle $i$      & \\       % & \si{m^2/s} \\
    $D_{p}$         & Particle diameter                         & \\       % & \si{m} \\
    $K_{ij}$        & Generalized coagulation coefficient       & \\        %& \si{m^3/s} \\
    $Kn$            & Knusden number                            & \\
    $k_B$           & Boltzmann's constant                      & \num{1.3806e-23} \si{J/K} \\
    $M_k$           & k$^{th}$ moment of the particle size distribution & \\  % & \si{kg^k/m^3} \\
    $MW_C$          & Molar mass of carbon                      & 12.011 \si{kg/kmol} \\
    $m_i$           & Mass of particle $i$                      & \\    % & \si{kg} \\
    $N_i$           & Number density of particles of size $i$ (per \si{kg} gas)  & \\   %& \si{\#/kg}\\
    $N_A$           & Avogadro's number                         & \num{6.0221e26} \\
    $P_{i}$         & Partial pressure of gas species $i$       & \\    % & \si{atm} \\
    $R_{coa}$       & Rate of particle coagulation              & \\    % & \si{m^{3}/\# s} \\
    $R_{nuc}$       & Rate of particle nucleation               & \\    % & \si{kmol/m^{3} s} \\
    $R_{grw}$       & Rate of particle surface growth           & \\    % & \si{kmol/m^{2} s} \\
    $R_{oxi}$       & Rate of particle oxidation                & \\    % & \si{kmol/m^{2} s} \\
    $R_{pi}$        & Radius of particle $i$                    & \\    % & \si{kmol/m^{2} s} \\
    $S$             & Particle surface area                     & \\    % & \si{m^2/m^3}-gas \\
    $T$             & Gas temperature                           & \\    % & \si{K} \\
    $Y_s$           & Soot mass fraction                        & \\    % & N/A \\
    $\alpha$             & Coagulation efficiency                    & \\
    $\beta_{c}$         & Continuum regime collision rate function      &  \\ % & \si{m^{3}/\# s} \\
    $\beta_{f}$         & Free-molecular regime collision rate function &  \\ % & \si{m^{3}/\# s} \\
    $\epsilon$             & Van der Waals enhancement factor          & 2.2 \\
    $\lambda_f$           & Gas mean free path                        &  \\ %      & \si{m} \\
    $\lambda_p$           & Particle mean free path                   &  \\ %      & \si{m} \\
    $\mu$             & Gas viscosity                             &  \\ % & \si{kg/m s} \\
    $\rho$             & Gas density                               &   \\ %     & \si{kg/m^3} \\
    $\rho_s$           & Solid soot density                        & 1850 \si{kg/m^3}  \\
    \hline
\end{tabularx}

}

%%%%%%%%%%%%%%%%%%%%%%%%%%%%%%%%%%%%%%%%%%%%%%%%%%%%%%%%%%%%%%%%%%%%%%%%%%%

\appendix

\section{Soot chemistry model rate details}

\subsection{Nagle and Strickland-Constable oxidation rate}
\label{a:NSC}

The rate expression for soot particle oxidation by \ce{O2} presented by Nagle and Strickland-Constable~\cite{Nagle_1962} is
\begin{equation}
    R_{oxi} = \rho_{\ce{C(s)}} \left[ k_A P_{\ce{O2}} \left( \frac{x}{1+k_Z P_{\ce{O2}}}\right) + k_B P_{\ce{O2}} (1-x) \right]
\end{equation}
where
\begin{equation}
    x=\frac{1}{1+\frac{k_T}{k_B P_{\ce{O2}}}}
\end{equation}
and
\begin{equation}
    k_A = 20e^{-15098/T}
\end{equation}
\begin{equation}
    k_B = \num{4.46e-3}e^{-7650/T}
\end{equation}
\begin{equation}
    k_T = \num{1.51e5}e^{-48817/T}
\end{equation}
\begin{equation}
    k_Z = 21.3e^{2063/T}
\end{equation}

\subsection{Fuchs generalized coagulation coefficient}
\label{a:FUCHS}
The Fuchs generalized coagulation coefficient~\cite{Fuchs_1964} for the collision between two particles takes the form $K_{12}=2\pi (D_{p1}+D_{p2})(D_1+D_2)\beta$, where
\begin{equation}
    \beta = \left[ \frac{D_{p1}+D_{p2}}{D_{p1}+D_{p2}+2(g_1^2+g_2^2)^{1/2}} + \frac{8(1/\alpha)(D_1+D_2)}{(\bar{c}_1^2+\bar{c}_2^2)^{1/2}(D_{p1}+D_{p2})} \right]^{-1}
\end{equation}
and 
\begin{align}
    \bar{c}_i &= \left( \frac{8k_B T}{\pi m_i} \right)^{1/2}, \\
    g_i &= \frac{\sqrt{2}}{3D_{pi}l_i} \left[ (D_{pi}+l_i)^3 - (D_{pi}^2+l_i^2)^{3/2} \right] - D_{pi}, \\
    l_i &= \frac{8D_i}{\pi \bar{c}_i}, \\
    D_i &= \frac{k_B T C_i}{3\pi \mu D_{pi}}.
\end{align}
In the continuum limit ($Kn \rightarrow 0$), $\beta=1$ and the generalized coagulation coefficient reduces to
\begin{equation}
    K_{12}=\frac{2k_BT}{3\mu} \frac{(D_{p1}+D_{p2})^2}{D_{p1}D_{p2}}.
\end{equation}
In the free-molecular limit ($Kn \rightarrow \infty)$, the coagulation coefficient can be reduced to
\begin{equation}
    K_{12} = \pi (R_{p1}+R_{p2})^2 (\bar{c}_1^2 + \bar{c}_2^2)^{1/2},
\end{equation}
which is equal to the expression obtained directly from kinetic theory~\cite{Seinfeld_2016}.

\subsection{Frenklach coagulation coefficient}
\label{a:FRENK}

The simplified transition regime model for the coagulation coefficient presented by Frenklach~\cite{Frenklach_2002b} also takes the form $K_{12}=2\pi (D_{p1}+D_{p2})(D_1+D_2)\beta$, but the collision factor $\beta$ is calculated as the harmonic mean of the continuum ($\beta_c$) and free-molecular ($\beta_f$) regime limit values,
\begin{equation}
    \beta = \frac{\beta_f \beta_c}{\beta_f + \beta_c},
\end{equation}
where
\begin{equation}
    \beta_c = \frac{2k_BT}{3 \mu} \left( \frac{C_1}{m_1^{1/3}} + \frac{C_2}{m_2^{1/3}} \right) (m_1^{1/3}+m_2^{1/3}),
\end{equation}
\begin{equation}
    \beta_f = \epsilon \sqrt{\frac{6k_BT}{\rho}} \left( \frac{3\pi \rho}{4} \right)^{1/6} \sqrt{\frac{1}{m_1}+\frac{1}{m_2}} (m_1^{1/3}+m_2^{1/3})^2,
\end{equation}
\begin{equation}
    C_i = 1 + 1.257Kn = 1 + 1.257 \frac{2\lambda_f}{D_i}.
\end{equation}

%%%%%%%%%%%%%%%%%%%%%%%%%%%%%%%%%%%%%%%%%%%%%%%%%%%%%%%%%%%%%%%%%%%%%%%%%%%

%% References:
%% If you have bibdatabase file and want bibtex to generate the
%% bibitems, please use
%%

\bibliographystyle{elsarticle-num}
\bibliography{sootlib-refs}

%% else use the following coding to input the bibitems directly in the
%% TeX file.

%\begin{thebibliography}{00}
%
%%% \bibitem{label}
%%% Text of bibliographic item
%\bibitem{Lignell_2018}
%D.~O. Lignell, V.~B. Lansinger, J.~Medina, M.~Klein, A.~R. Kerstein,
%H.~Schmidt, M.~Fistler, M.~Oevermann, One-dimensional turbulence modeling for
%cylindrical and spherical flows: model formulation and application,
%Theoretical and Computational Fluid Dynamics 32~(4) (2018) 495--520.
%\newblock \href {http://dx.doi.org/10.1007/s00162-018-0465-1}
%{\path{doi:10.1007/s00162-018-0465-1}}.
%
%\end{thebibliography}

%%%%%%%%%%%%%%%%%%%%%%%%%%%%%%%%%%%%%%%%%%%%%%%%%%%%%%%%%%%%%%%%%%%%%%%%%%%

\section*{Required Metadata}

\section*{Current code version}

Ancillary data table required for subversion of the codebase. Kindly replace examples in right column with the correct information about your current code, and leave the left column as it is.

\begin{table}
\begin{tabular}{|l|p{6.5cm}|p{6.5cm}|}
\hline
\textbf{Nr.} & \textbf{Code metadata description} & \textbf{Please fill in this column} \\
\hline
C1 & Current code version & 1.0 \\
\hline
C2 & Permanent link to code/repository used for this code version & https://github.com/byuignite/sootlib \\
\hline
C3  & Permanent link to Reproducible Capsule & \\
\hline
C4 & Legal Code License & MIT \\
\hline
C5 & Code versioning system used & Git \\
\hline
C6 & Software code languages, tools, and services used & C++ \\
\hline
C7 & Compilation requirements, operating environments \& dependencies & C++11, CMake 3.15+, Catch2 (optional) \\
\hline
C8 & If available Link to developer documentation/manual &  \\
\hline
C9 & Support email for questions & davidlignell@byu.edu \\
\hline
\end{tabular}
\caption{Code metadata (mandatory)}
\end{table}

%\section*{Current executable software version}
%
%Ancillary data table required for sub version of the executable software: (x.1, x.2 etc.) kindly replace examples in right column with the correct information about your executables, and leave the left column as it is.
%
%\begin{table}
%\begin{tabular}{|l|p{6.5cm}|p{6.5cm}|}
%\hline
%\textbf{Nr.} & \textbf{(Executable) software metadata description} & \textbf{Please fill in this column} \\
%\hline
%S1 & Current software version & 1.0 \\
%\hline
%S2 & Permanent link to executables of this version  &  \\
%\hline
%S3  & Permanent link to Reproducible Capsule & \\
%\hline
%S4 & Legal Software License & MIT \\
%\hline
%S5 & Computing platforms/Operating Systems & Windows, MacOS, Linux \\
%\hline
%S6 & Installation requirements \& dependencies & C++11, CMake 3.15+, Catch2 (optional)\\
%\hline
%S7 & If available, link to user manual - if formally published include a reference to the publication in the reference list & \\
%\hline
%S8 & Support email for questions & davidlignell@byu.edu \\
%\hline
%\end{tabular}
%\caption{Software metadata (optional)}
%\end{table}

\end{document}
\endinput
%%
%% End of file `SoftwareX_article_template.tex'.