%! suppress = MathOperatorEscape
%%
%% Copyright 2007, 2008, 2009 Elsevier Ltd
%% 
%% This file is part of the 'Elsarticle Bundle'.
%% ---------------------------------------------
%% 
%% It may be distributed under the conditions of the LaTeX Project Public
%% License, either version 1.2 of this license or (at your option) any
%% later version.  The latest version of this license is in
%%    http://www.latex-project.org/lppl.txt
%% and version 1.2 or later is part of all distributions of LaTeX
%% version 1999/12/01 or later.
%% 
%% The list of all files belonging to the 'Elsarticle Bundle' is
%% given in the file `manifest.txt'.
%% 

%% Template article for Elsevier's document class `elsarticle'
%% with numbered style bibliographic references
%% SP 2008/03/01

\documentclass[preprint,12pt,letterpaper]{elsarticle}

%% Use the option review to obtain double line spacing
%% \documentclass[authoryear,preprint,review,12pt]{elsarticle}

%-------- packages --------
\usepackage{amsmath}
\usepackage{amssymb}
\usepackage{hyperref}
\usepackage{lineno}                 % line numbers
\usepackage[version=4]{mhchem}      % chem formatting
\usepackage{siunitx}                % units formatting
\usepackage{rotating}

%-------- package setup --------
\sisetup{exponent-mode = scientific}

%-------- front matter --------
\journal{SoftwareX}

\begin{document}

\begin{frontmatter}

%% Title, authors and addresses

%% use the tnoteref command within \title for footnotes;
%% use the tnotetext command for theassociated footnote;
%% use the fnref command within \author or \address for footnotes;
%% use the fntext command for theassociated footnote;
%% use the corref command within \author for corresponding author footnotes;
%% use the cortext command for theassociated footnote;
%% use the ead command for the email address,
%% and the form \ead[url] for the home page:
%% \title{Title\tnoteref{label1}}
%% \tnotetext[label1]{}
%% \author{Name\corref{cor1}\fnref{label2}}
%% \ead{email address}
%% \ead[url]{home page}
%% \fntext[label2]{}
%% \cortext[cor1]{}
%% \address{Address\fnref{label3}}
%% \fntext[label3]{}

\title{SootLib: a soot model library for combustion CFD}

%% use optional labels to link authors explicitly to addresses:
%% \author[label1,label2]{}
%% \address[label1]{}
%% \address[label2]{}

%\renewcommand{\thefootnote}{\fnsymbol{footnote}}
\author{Victoria B. Stephens}
\author{Josh Bedwell}
\author{David O. Lignell\corref{cor1}}

\cortext[cor1]{Corresponding author \ead{davidlignell@byu.edu}}

\address{Chemical Engineering Department, Brigham Young University, Provo, UT 84602, USA}

\begin{abstract}
sootlib abstract goes here
\end{abstract}

\begin{keyword}
soot \sep combustion \sep reacting flow simulation
\end{keyword}

\end{frontmatter}

\linenumbers

%-------- main text --------

%%%%%%%%%%%%%%%%%%%%%%%%%%%%%%%%%%%%%%%%%%%%%%%%%%%%%%%%%%%%%%%%%%%%%%%%%%%

\section{Introduction}
\label{s:intro}

Most commercial combustion processes that produce energy involve turbulent, non-premixed flames, which produce soot. Soot is responsible for a flame's luminosity, generates a large portion of a flame's radiative heat transfer to its surroundings, and contributes to many of the health, safety, and environmental hazards associated with air pollution from combustion systems~\cite{EPA_2009,EPA_2004}. In order to address soot's negative effects and optimize practical combustion processes, scientists and engineers require a better understanding of soot's fundamental structure and behavior in combustion environments.

Because combustion processes are so complex, modeling and simulation cannot be fully separated from studying fundamental chemical processes; instead, we must use incomplete knowledge and imperfect models to investigate both simultaneously. We often rely on experimental data for comparison, and as a result, errors must be distinguished by their source\textemdash experimental, theoretical, or computational\textemdash which further complicates modeling. Turbulent combustion uniquely challenges modelers because it involves tightly coupled equations of multicomponent mass transfer, convective and radiative heat transfer, turbulent fluid dynamics, multi-phase flow, and complex chemical kinetics. These span many orders of magnitude in both their length and time scales. Direct simulation approaches can produce accurate simulation data by resolving the full range of length and time scales, but the computational cost can be prohibitively high, particularly for simulating practical combustion processes of interest to engineers~\cite{Pope_2000,Bernard_2002}. Simulating soot in flames further expands the range of scales that must be considered, adding additional complexity and computational cost to direct simulations.

High computational cost in combustion simulations is often addressed by modeling various aspects of the configuration. In particular, computational models that quantify soot production in simulations can help us study its fundamental behavior and distinguish between various reaction mechanisms and transport models. Reactions involving soot are not purely chemical, but also involve particle aggregation, size distributions, and transport that may or may not affect molecular reactions and heat transfer within a flame. Accurate models and simulations that help us clarify the fundamental processes that control soot formation and transport in flames represent an important step forward in the study of combustion systems~\cite{Frenklach_2002b}.

With SootLib, we provide a convenient, easy-to-use access point for soot property and particle dynamics models that can be interfaced with various simulation approaches for combustion CFD. SootLib is a modular C++ library that can compute various source terms for the reacting Navier Stokes equations. Its modular design makes model parts interchangeable (within limits), allowing users to quickly and easily compare and contrast models. Several [TODO: put a number here] models are fully implemented and validated and use a common interface, allowing researchers convenient access to soot modeling tools suitable for various reacting CFD applications.

%%%%%%%%%%%%%%%%%%%%%%%%%%%%%%%%%%%%%%%%%%%%%%%%%%%%%%%%%%%%%%%%%%%%%%%%%%%

\section{Model Descriptions}
\label{s:models}

Soot models can generally be broken down into two interrelated parts: chemistry and particle size distribution (PSD). Soot chemistry models address the chemical reactions involved in soot behavior, while PSD models describe the size distribution of soot particles. Given a specific state and a set of soot chemistry mechanisms, the PSD model predicts the amounts and relative locations of different sized soot particles based on their current distribution, the desired chemical mechanisms, and the thermodynamic state of the surrounding gas. Using a particular set of soot chemistry models does not necessitate using a particular PSD model, and vice versa. SootLib takes advantage of this distinction by allowing users to specify individual models rather than predetermined model sets, giving users more flexibility and model developers more control in soot simulations.

%--------------------------------------------------------------------------
\subsection{Chemistry}
\label{ss:chemistry}

Soot chemistry is typically divided into four main types: nucleation reactions describe how soot particles form from gaseous precursors and may or may not include the condensation of large polycyclic aromatic hydrocarbons (PAH); surface growth refers to particle growth by the chemical addition of gaseous species to existing soot particles; oxidation reactions describe soot particle size reduction by chemical reactions, usually with oxygen-based gaseous species; and coagulation defines the rate at which soot particles combine to form new particles (and may or may not include the aggregation of particles into large fractal soot structures). A complete soot chemistry model typically provides a mechanism for each of the four major steps, each of which occurs independently but all of which depend heavily on the composition and thermodynamic properties of the surrounding gas. Some models provide all four mechanisms, while later research tends to expand on earlier mechanisms or focus on one type. SootLib collects a range of models from the literature and implements them with a uniform interface, offering model developers a flexible testing method and giving research simulations a consistent framework for parametric simulations. Table~\ref{t:chem_models} summarizes the soot chemistry models implemented in SootLib;  Table~\ref{t:variables} defines the variables used in the presented models.

In the following models, soot is defined as a collection of carbon atoms with lesser amounts of hydrogen and other elements. Soot particles are usually assumed to be spherical for purposes of calculating particle diameter and surface area; this is a reasonable assumption for nascent and relatively small soot particles, but may not be adequate to describe large soot agglomerates, which tend to exhibit fractal structures~\cite{Fuchs_1964}.

Particle nucleation refers to the mechanisms by which the smallest possible soot particles are formed. The most common definition used in soot modeling literature defines a soot particle as having at least 100 carbon atoms~\cite{Leung_1991}; this is the value used by SootLib for kinetic-style nucleation mechanisms. The minimum soot particle size may also be defined by the nucleation mechanism itself, such as in SootLib's PAH nucleation mechansism, which defines the smallest soot particles as the result of a collsion between two PAH dimers. The simplest nucleation models link nucleation rate to the concentration of acetylene (\ce{C2H2}) in the gas mixture, while more complex models may account for any number of gaseous hydrocarbon species.

Surface growth refers to the addition of carbon atoms to an existing soot particle from gaseous hydrocarbons. Simple surface growth models typically rely on acetylene (\ce{C2H2}) as the primary source of gaseous carbon, though other gaseous hydrocarbons may also contribute to particle surface growth to varying degrees. Additionally, surface growth models tend to include some dependence on the soot particle surface area since the availability of sites for addition of carbon atoms to existing particles tends to be a limiting factor in rate calculations.

Soot oxidation refers to the process by which soot particles decrease in size due to reactions with gaseous species. Similar to surface growth, oxidation mechanisms may depend on the available surface area of oxidizing soot particles, which may or may not be a limiting factor depending on the thermodynamic state, composition of the gas mixture, and the amount of soot present. Early soot oxidation models tend to rely on \ce{O2} as the principal oxidant, though experimental studies show that \ce{OH} and sometimes \ce{O} can also contribute significantly to soot oxidation rates.

Coagulation refers to the process by which soot particles increase in size due to collisions with other soot particles. It is closely related to the soot particle size distribution (PSD), which describes the population of soot particles in terms of size, mass, and number density. While most coagulation processes are not technically chemical reactions, coagulation models are typically structured so as to be analagous to the strictly chemical steps and are usually categorized as a type of soot chemistry. At present, SootLib does not include any coagulation models that account for the agglomeration of soot particles into large fractal aggregates.

\begin{sidewaystable}
    \caption{Summary of soot chemistry models implemented in SootLib. In mechanisms, \ce{C(s)} represents a soot particle, i.e. "solid" carbon, and \ce{C(s)^.} indicates a radical site on a soot particle, typically caused by hydrogen abstraction. Variable definitions for rate expressions can be found in Table~\ref{t:variables}. Rate expressions that involve multiple expressions or otherwise do not fit well in this table can be found in the appendices as noted.}
    \label{t:chem_models}
    \centering
    \resizebox{\textwidth}{!}{
        \begin{tabular}{l l l l l}
            \hline
            Chemistry type  & Model                 & Model ID     & Mechanism & Rate expression \\
            \hline \hline
            Nucleation      & Leung \& Lindstedt~\cite{Leung_1991}  & \texttt{LL}    & \ce{C2H2 -> 2C(s) + H2} & $R_{nuc} = \num{0.1e5} e^{-21100/T} \ce{[C2H2]}$\\
                            & Lindstedt 2005~\cite{Lindstedt_2005}  & \texttt{LIN}   & \ce{C2H2 -> 2C(s) + H2} & $R_{nuc} = \num{0.63e4} e^{-21100/T} \ce{[C2H2]}$\\
                            & PAH nucleation~\cite{Blanquart_2009}  & \texttt{PAH}   & dimerization of gaseous PAH  &\\
            \hline
            Surface growth  & Leung \& Lindstedt~\cite{Leung_1991}  & \texttt{LL}    & \ce{C2H2 + nC(s) ->} (n+2)\ce{C(s) + H2} & $R_{grw} = \num{0.6e4} e^{-21100/T} f(S) \ce{[C2H2]}$\\
                & Lindstedt 1994~\cite{Lindstedt_1994}  & \texttt{LIN}   & \ce{C2H2 + nC(s) ->} (n+2)\ce{C(s) + H2} & $R_{grw} = \num{0.1e-11} e^{-12100/T} \ce{[C2H2]} 2M_0 MW_{C}$ \\
                & HACA~\cite{Appel_2000,Frenklach_1994} & \texttt{HACA}  & \ce{C(s)-H + H <=> C(s)^. + H2}   & $R_{grw,f}=\num{4.2e13} e^{-13/RT} \ce{[H]}$ \\
                &                                       &                &                                   & $R_{grw,r}=\num{3.9e12} e^{-11/RT} \ce{[H2]}$ \\
                &                                       &                & \ce{C(s)^. + H -> C(s)-H}         & $R_{grw}=\num{2.0e13} \ce{[H]}$ \\
                &                                       &                & \ce{C(s)^. + C2H2 -> C(s)-H + H}  & $R_{grw}=\num{8.0e7} T^{1.56} e^{-3.8/RT} \ce{[C2H2]}$\\
                &                                       &                & \ce{C(s)-H + OH <=> C(s)^. + H2O} & $R_{grw,f}=\num{1e10} T^{0.734} e^{-1.43/RT} \ce{[OH]}$ \\
                &                                       &                &                                   & $R_{grw,r}=\num{3.68e8} T^{1.139} e^{-17.1/RT} \ce{[H2O]}$ \\

            \hline
            Oxidation       & Leung \& Lindstedt~\cite{Leung_1991}   & \texttt{LL}   &  \ce{C(s) + 1/2O2 -> CO} & $R_{oxi,\ce{O2}} = \num{0.1e5} T^{1/2} e^{-19680/T} f(S) \ce{[O2]}$\\
                            & HACA~\cite{Appel_2000,Frenklach_1994} & \texttt{HACA}  & \ce{C(s)^. + O2 -> 2CO + products} & $R_{oxi,\ce{O2}}=\num{2.2e12} e^{-7.5/RT} \ce{[O2]}$\\
                            &                                       &                & \ce{C(s)-H + OH -> CO + products} & $R_{oxi,\ce{OH}}=0.13*1290 P_{\ce{OH}} T^{-1/2} $\\
                            & Lee~\cite{Lee_1962} +
                              Neoh~\cite{Neoh_1980,Neoh_1981}       & \texttt{LEE\textunderscore NEOH} & \ce{C + 1/2O2 -> CO} & $R_{oxi,\ce{O2}} = \num{1.085e4} P_{\ce{O2}} T^{-1/2} e^{-19778.24/T}$\\
                            &                                       &                & \ce{C + OH -> CO + H} & $R_{oxi,\ce{OH}}=0.13*1290 P_{\ce{OH}} T^{-1/2}$ \\
                            & NSC~\cite{Nagle_1962} +
                              Neoh~\cite{Neoh_1980,Neoh_1981}       & \texttt{NSC\textunderscore NEOH} & \ce{C + 1/2O2 -> CO} & see~\ref{a:NSC}\\
                            &                                       &                & \ce{C + OH -> CO + H} & $R_{oxi,\ce{OH}}=0.13*1290 P_{\ce{OH}} T^{-1/2}$\\
            \hline
            Coagulation     & Leung \& Lindstedt~\cite{Leung_1991}  & \texttt{LL}    & \ce{nC(s) -> C_n(s)} & $R_{coa} = -2C_a d_p^{1/2} \left( \frac{6k_B T}{\rho_{C(s)}}\right) (\rho N)^2$\\
                            & Fuchs~\cite{Fuchs_1964,Seinfeld_2016} & \texttt{FUCHS} & \ce{nC(s) -> C_n(s)} & $K_{12}=2\pi (D_{p1}+D_{p2})(D_1+D_2)\beta$ \\
                            & Frenklach~\cite{Frenklach_2002}       & \texttt{FRENK} & \ce{nC(s) -> C_n(s)} & $K_{12}=\beta_{FM}\beta_{C}/(\beta_{FM}+\beta_{C})$ \\
            \hline
        \end{tabular}
    }
\end{sidewaystable}

\begin{table}
    \caption{Variables used in soot chemistry rate expressions.}
    \label{t:variables}
    \centering
    \resizebox{\textwidth}{!}{
        \begin{tabular}{l l l l}
            \hline
            Variable        & Definition                               & Value & Units \\
            \hline \hline
            $C_a$           & Agglomeration rate constant              & 9.0    & N/A \\
            $C_{min}$       & Number of carbon atoms in an incipient soot particle & 100 & N/A \\
            \ce{[C2H2]}     & Concentration of gaseous acetylene       &        & \si{kmol/m^3} \\
            \ce{[O2]}       & Concentration of gaseous oxygen          &        & \si{kmol/m^3} \\
            $d_p$           & Soot particle diameter                   &        & \si{m} \\
            $K_{12}$        & Generalized coagulation coefficient applied to two particles & &  \\
            $M_k$           & k$^{th}$ moment of the soot particle size distribution &  & \si{kg^k/m^3} \\
            $MW_{C}$        & Molar mass of carbon                     & 12.011 & \si{kg/kmol} \\
            $N$             & Number of soot particles per \si{kg} gas &        & N/A \\
            $N_A$           & Avogadro's number                        & \num{6.0221e26} & N/A \\
            $P_{i}$         & Partial pressure of gaseous species $i$  &        & \si{atm} \\
            $R_{nuc}$       & Rate of particle nucleation              &        & \si{kmol/m^{3} s} \\
            $R_{grw}$       & Rate of particle surface growth          &        & \si{kmol/m^{2} s} \\
%            $R_{grw,f}$     & Rate of particle surface growth, forward reaction &        & \si{kmol/m^{2} s} \\
%            $R_{grw,r}$     & Rate of particle surface growth, reverse reaction &        & \si{kmol/m^{2} s} \\
            $R_{oxi}$       & Rate of particle oxidation               &        & \si{kmol/m^{2} s} \\
            $R_{coa}$       & Rate of particle coagulation             &        & \si{m^{3}/\# s} \\
            $S$             & Particle surface area                    &        & \si{m^2/m^3}-gas \\
            $T$             & Gas temperature                          &        & \si{K} \\
            $Y_{C(S)}$      & Mass fraction of soot                    &        & N/A \\
            $\beta_{C}$     & Continuum regime collision rate function      &        & \si{m^{3}/\# s} \\
            $\beta_{FM}$    & Free-molecular regime collision rate function &        & \si{m^{3}/\# s} \\
            $k_B$        & Boltzmann's constant                     & \num{1.3806e-23} & \si{J/K} \\
            $\rho$          & Gas density                              &        & \si{kg/m^3} \\
            $\rho_{C(s)}$   & Solid soot particle density              & 1850   & \si{kg/m^3} \\
            \hline
        \end{tabular}
    }
\end{table}

\subsubsection{Global kinetic mechanisms}
\label{sss:global}

Global kinetic mechanisms, typically represented by Arrhenius-style rate expressions, are popular choices in combustion simulations involving soot, particularly in cases that require soot modeling but cannot afford the relatively high computational cost of detailed soot models or gas mechanisms. Because global models represent each type of soot chemistry with simplified and often empirical rate expressions, they do not capture all of the fundamental mechanisms of various soot phenomena, but they have the advantages of being relatively simple, easy to implement, and computationally inexpensive. In other words, global models tend to sacrifice accuracy in favor of high speed and low computational cost. Additionally, commonly used global models can serve as convenient points of reference when developing or testing modified or more complex soot models.

SootLib's simplest chemistry model is a simplified kinetic mechanism presented by Leung and Lindstedt (\texttt{LL}) that consists of four Arrhenius-style rate expressions: one each for soot nucleation, surface growth, oxidation, and coagulation~\cite{Leung_1991}; refer to  Table~\ref{t:chem_models} for the complete rate expressions. Note that the surface growth and oxidation steps include a function $f(S)$ representing the surface growth and oxidation rates' dependence on soot particle surface area; specifically, it is assumed that the number of local active sites available for surface growth and oxidation reactions is proportional to the square root of the total particle surface area in the flame. If the surface area is given by $S=\pi d_p^2 \rho N$ and the soot particle diameter by $d_p=(6Y_{C(s)}/\pi \rho_{C(s)} N)^{1/3}$, this surface area dependence becomes
\begin{equation}
    \label{e:surface_area}
    f(S) = \sqrt{\pi \left( \frac{6MW_{C}}{\pi \rho_{C(s)}} \right) ^{2/3}} \left[ \frac{\rho Y_{C(s)}}{MW_{C(s)}} \right]^{1/3} [\rho N]^{1/6}.
\end{equation}
The Leung and Lindstedt model is included with SootLib as a point of reference because it is one of the most common global soot models used in combustion simulations. This model was developed using observations of laboratory-scale ethylene jet flames, and its accuracy and applicability are generally limited to conditions similar to those under which it was developed.

Various modifications have been proposed to address the inaccuracies of the Leung and Lindstedt model. Lindstedt proposed an alteration to the Leung and Lindstedt nucleation step's preexponential factor to increase accuracy without changing the expression's form~\cite{Lindstedt_2005}. Lindstedt also proposed a modified expression for particle surface growth that eliminates its dependence on particle surface area in favor of dependence on the particle number density~\cite{Lindstedt_1994}. In SootLib, these models can be accessed by specifying \texttt{LIN} for the nucleation and surface growth chemistry, respectively.

The Leung and Lindstedt oxidation expression is itself based on an earlier model presented by Lee et al.~\cite{Lee_1962}, which models the oxidation of soot particles by \ce{O2} with a global rate expression (see Table~\ref{t:chem_models}). Nagle and Strickland-Constable~\cite{Nagle_1962} also presented a commonly used model for soot oxidation by \ce{O2} in which the rate of oxidation depends on a nonlinear combination of Arrhenius-style rate constants applied to the partial pressure of \ce{O2}; see~\ref{a:NSC} for the full rate expression. These early soot oxidation models, however, do not take into account the influence of oxidation by \ce{OH}, which can be significant for certain flame configurations and fuels~\cite{Neoh_1980}. To account for this, SootLib adds expressions for \ce{OH} oxidation presented by Neoh et al.~\cite{Neoh_1981} to the oxidation expressions presented by Lee et al. and Nagle and Strickland-Constable, resulting in the \texttt{LEE\textunderscore NEOH} and \texttt{NSC\textunderscore NEOH} options for soot oxidation in SootLib. Leung and Lindstedt explicitly acknowledge the lack of oxidation by \ce{OH} in their model, but consider their one-step oxidation mechanism sufficient for their purposes; in light of their reasoning and the widespread use of the Leung and Lindstedt soot model as published, SootLib does not alter or add to the existing Leung and Lindstedt oxidation step.

\subsubsection{Physics-based models}
\label{sss:physics-models}

Physics-based models and mechanisms attempt to increased the accuracy of more empirical models by using multiple elementary reaction steps to represent actual soot behavior rather than relying on the empiricism inherent in global reaction models. By accounting for fundamental phenomena, physics-based models may result in increased accuracy at the cost of computational speed and efficiency. The two-step oxidation models presented above are simple examples of how using multiple reaction steps can increase accuracy; by accounting for oxidation by both \ce{O2} and \ce{OH}, such models can use validated rate data for elementary reaction steps, potentially resulting in higher accuracy under certain coonditions than a model that only considers one oxidation species or combines both dependencies into one empirical rate expression.

The hydrogen-abstraction acetylene-addition (HACA) mechanism is a popular multi-step model for soot surface growth and oxidation~\cite{Appel_2000}. It breaks down the global dependence of surface growth on acetylene into elementary reaction steps, each with its own Arrhenius-style rate expression. This approach potentially increases the accuracy of the model by reducing the amount of empiricism in favor of a physics-based reaction mechanism.

[HACA FIGURE HERE]

Another compelling physics-based model for soot particle nucleation considers the inception of soot particles due to the collision of polycyclic aromatic hydrocarbons (PAH), as presented by Blanquart and Pitsch~\cite{Blanquart_2009c}. In it, incipient soot particles are created by the collision of two PAH dimers, which are themselves created by the collision of two PAH molecules. This overrides the baseline used by global models, which define soot particles as having at least 100 carbon atoms, but is much closer to reality, in which all chemical species, soot particles included, exist on a spectrum and behave according to their size and mass. Experimental literature indicates that soot produced by gaseous fuels tends to nucleate via PAH collisions, though the exact mechanisms are unknown and depend on the fuel, its surrounding environment, and various other factors [CITATION]. The present PAH nucleation model also accounts for condensation of PAH dimers onto existing soot particles, which represents an additional surface growth mechanism.

[PAH NUCLEATION/CONDENSATION FIGURE HERE]

\subsubsection{Coagulation mechanisms}
\label{sss:coagulation}

SootLib includes three coagulation mechanisms, all of which interact strongly with the particle size distribution (PSD) model, discussed in Section~\ref{ss:PSD_dynamics}. The simplest coagulation model is that presented as part of the Leung and Lindstedt model (\texttt{LL}), which describes particle coagulation with a normal square dependence using an empirical agglomeration rate coefficient~\cite{Leung_1991}.

Physics-based coagulation models represent the coagulation rate of two particles with the expression $R_{coa,12}=K_{12}N_1N_2$, where $N_1$ and $N_2$ are the number density of particles 1 and 2, respectively. $K_{12}$ is coagulation coefficient given by $K_{12}=2\pi (D_{p1}+D_{p2})(D_1+D_2)$, where $D_{p1}$ and $D_{p2}$ are the diameters and $D_1$ and $D_2$ are the Brownian diffusivites of particles 1 and 2, respectively~\cite{Seinfeld_2016}. Particle diffusivity, in turn, depends on the surrounding conditions. In the continuum regime, when the mean free path $\lambda_p$ of a diffusing particle is comparable to its radius, its diffusion can be described by the Einstein-Stokes relation; in the free-molecular regime, when the mean free path $\lambda_p$ of the particle is much larger than its radius, diffusion can be described with kinetic theory. Fuchs proposed a generalized coagulation coefficient of the form $K_{12}=2\pi (D_{p1}+D_{p2})(D_1+D_2)\beta$, where $\beta$ is a correction factor that accounts for the kinetic effect of the particle regime~\cite{Fuchs_1964, Seinfeld_2016}.

%--------------------------------------------------------------------------
\subsection{Particle size distribution and dynamics}
\label{ss:PSD_dynamics}

%In addition to the chemical reactions that influence their creation and destruction, soot particles

% General info about number density functions and using moments to describe a PSD.
% Maybe include derivation of moment-based NDF source terms

\begin{table}
    \caption{Summary of soot particle size distribution models implemented in SootLib.}
    \label{t:psd_models}
    \centering
    \resizebox{\textwidth}{!}{
        \begin{tabular}{l l l}
            \hline
            PSD model                                    & Model ID & \# Moments   \\
            \hline
            Assumed monodisperse~\cite{Lignell_2008b}     & MONO  & 2      \\
            Assumed lognormal~\cite{Lignell_2008b}        & LOGN  & 3  \\
            Quadrature method of moments~\cite{McGraw_1997} & QMOM  & 2, 4, 6  \\
            Method of moments with interpolative closure~\cite{Frenklach_2002b} & MOMIC & 2\textendash8  \\
            \hline
        \end{tabular}
    }
\end{table}

\subsubsection{Monodisperse distribution (MONO)}
\label{sss:mono}

\subsubsection{Lognormal distribution (LOGN)}
\label{sss:logn}

\subsubsection{Quadrature method of moments (QMOM)}
\label{sss:qmom}

\subsubsection{Method of moments with interpolative closure (MOMIC)}
\label{sss:momic}

%\subsubsection{Sectional model (SECT)}
%\label{sss:sect}

\subsection{Model combinations and limitations}
\label{ss:limitations}

%%%%%%%%%%%%%%%%%%%%%%%%%%%%%%%%%%%%%%%%%%%%%%%%%%%%%%%%%%%%%%%%%%%%%%%%%%%

\section{Software description}
\label{s:architecture}

%Describe the software in as much as is necessary to establish a vocabulary needed to explain its impact.
%
%Give a short overview of the overall software architecture; provide a pictorial component overview or similar (if possible). If necessary provide implementation details.
%
%Present the major functionalities of the software.

%%%%%%%%%%%%%%%%%%%%%%%%%%%%%%%%%%%%%%%%%%%%%%%%%%%%%%%%%%%%%%%%%%%%%%%%%%%

\section{Validation and Examples}
\label{s:examples}

%Provide at least one illustrative example to demonstrate the major functions.

%%%%%%%%%%%%%%%%%%%%%%%%%%%%%%%%%%%%%%%%%%%%%%%%%%%%%%%%%%%%%%%%%%%%%%%%%%%

\section{Discussion}
\label{s:discussion}

%\textbf{This is the main section of the article and the reviewers weight the description here appropriately}
%
%Indicate in what way new research questions can be pursued as a result of the software (if any).
%
%Indicate in what way, and to what extent, the pursuit of existing research questions is improved (if so).
%
%Indicate in what way the software has changed the daily practice of its users (if so).
%
%Indicate how widespread the use of the software is within and outside the intended user group.
%
%Indicate in what way the software is used in commercial settings and/or how it led to the creation of spin-off companies (if so).

%%%%%%%%%%%%%%%%%%%%%%%%%%%%%%%%%%%%%%%%%%%%%%%%%%%%%%%%%%%%%%%%%%%%%%%%%%%

\section{Conclusions}
\label{s:conclusions}

Set out the conclusion of this original software publication.

%%%%%%%%%%%%%%%%%%%%%%%%%%%%%%%%%%%%%%%%%%%%%%%%%%%%%%%%%%%%%%%%%%%%%%%%%%%

\section{Conflict of Interest}
%Please select the appropriate text:

%Potential conflict of interest exists:
%We wish to draw the attention of the Editor to the following facts, which may be considered as potential conflicts of interest, and to significant financial contributions to this work. The nature of potential conflict of interest is described below: [Describe conflict of interest]

%No conflict of interest exists:
The authors declare that they have no known competing financial interests or personal relationships that could have appeared to influence the work reported in this paper.

%%%%%%%%%%%%%%%%%%%%%%%%%%%%%%%%%%%%%%%%%%%%%%%%%%%%%%%%%%%%%%%%%%%%%%%%%%%

\section*{Acknowledgements}

%The authors extend special thanks to Hadi Bordbar for assistance with the WSGG model and to Vladimir Solovjov and Brent Webb for their insights and assistance with the RCSLW model.
This research did not receive any specific grant from funding agencies in the public, commercial, or not-for-profit sectors.

%%%%%%%%%%%%%%%%%%%%%%%%%%%%%%%%%%%%%%%%%%%%%%%%%%%%%%%%%%%%%%%%%%%%%%%%%%%

%% References:
%% If you have bibdatabase file and want bibtex to generate the
%% bibitems, please use
%%

\bibliographystyle{elsarticle-num}
\bibliography{sootlib-refs}

%% else use the following coding to input the bibitems directly in the
%% TeX file.

%\begin{thebibliography}{00}
%
%%% \bibitem{label}
%%% Text of bibliographic item
%\bibitem{Lignell_2018}
%D.~O. Lignell, V.~B. Lansinger, J.~Medina, M.~Klein, A.~R. Kerstein,
%H.~Schmidt, M.~Fistler, M.~Oevermann, One-dimensional turbulence modeling for
%cylindrical and spherical flows: model formulation and application,
%Theoretical and Computational Fluid Dynamics 32~(4) (2018) 495--520.
%\newblock \href {http://dx.doi.org/10.1007/s00162-018-0465-1}
%{\path{doi:10.1007/s00162-018-0465-1}}.
%
%
%\end{thebibliography}

\appendix

\section{Soot chemistry model rate details}

\subsection{Nagle and Strickland-Constable oxidation rate}
\label{a:NSC}

The rate expression for soot particle oxidation by \ce{O2} presented by Nagle and Strickland-Constable~\cite{Nagle_1962} is
\begin{equation}
    R_{oxi} = \rho_{\ce{C(s)}} \left[ k_A P_{\ce{O2}} \left( \frac{x}{1+k_Z P_{\ce{O2}}}\right) + k_B P_{\ce{O2}} (1-x) \right]
\end{equation}
where
\begin{equation}
    x=\frac{1}{1+\frac{k_T}{k_B P_{\ce{O2}}}}
\end{equation}
and
\begin{equation}
    k_A = 20e^{-15098/T}
\end{equation}
\begin{equation}
    k_B = \num{4.46e-3}e^{-7650/T}
\end{equation}
\begin{equation}
    k_T = \num{1.51e5}e^{-48817/T}
\end{equation}
\begin{equation}
    k_Z = 21.3e^{2063/T}
\end{equation}

\subsection{Fuchs generalized coagulation coefficient}
\label{a:FUCHS}
The Fuchs generalized coagulation coefficient applied to two particles $K_{12}$ takes the form $K_{12}=2\pi (D_{p1}+D_{p2})(D_1+D_2)\beta$, where
\begin{equation}
    \beta = \left[ \frac{D_{p1}+D_{p2}}{D_{p1}+D_{p2}+2(g_1^2+g_2^2)^{1/2}} + \frac{8(1/\alpha)(D_1+D_2)}{(\={c}_1^2+\={c}_2^2)^{1/2}(D_{p1}+D_{p2})} \right]^{-1}
\end{equation}
and 
\begin{align}
    \={c}_i &= \left( \frac{8k_B T}{\pi m_i} \right)^{1/2} \\
    g_i &= \frac{\sqrt{2}}{3D_{pi}\lambda_i} \left[ (D_{pi}+\lambda_i)^3 - (D_{pi}^2+\lambda_i^2)^{3/2} \right] - D_{pi} \\
    \lambda_i &= \frac{8D_i}{\pi \={c}_i} \\
    D_i &= \frac{k_B T C_c}{3\pi \mu D_{pi}}
\end{align}

\subsection{Frenklach modified coagulation coefficient}
\label{a:FRENK}

\section*{Required Metadata}

\section*{Current code version}

Ancillary data table required for subversion of the codebase. Kindly replace examples in right column with the correct information about your current code, and leave the left column as it is.

\begin{table}[!h]
\begin{tabular}{|l|p{6.5cm}|p{6.5cm}|}
\hline
\textbf{Nr.} & \textbf{Code metadata description} & \textbf{Please fill in this column} \\
\hline
C1 & Current code version & For example v42 \\
\hline
C2 & Permanent link to code/repository used for this code version & For example: $https://github.com/mozart/mozart2$ \\
\hline
C3  & Permanent link to Reproducible Capsule & \\
\hline
C4 & Legal Code License   & List one of the approved licenses \\
\hline
C5 & Code versioning system used & For example svn, git, mercurial, etc. put none if none \\
\hline
C6 & Software code languages, tools, and services used & For example C++, python, r, MPI, OpenCL, etc. \\
\hline
C7 & Compilation requirements, operating environments \& dependencies & \\
\hline
C8 & If available Link to developer documentation/manual & For example: $http://mozart.github.io/documentation/$ \\
\hline
C9 & Support email for questions & \\
\hline
\end{tabular}
\caption{Code metadata (mandatory)}
\end{table}

\section*{Current executable software version}

Ancillary data table required for sub version of the executable software: (x.1, x.2 etc.) kindly replace examples in right column with the correct information about your executables, and leave the left column as it is.

\begin{table}[!h]
\begin{tabular}{|l|p{6.5cm}|p{6.5cm}|}
\hline
\textbf{Nr.} & \textbf{(Executable) software metadata description} & \textbf{Please fill in this column} \\
\hline
S1 & Current software version & For example 1.1, 2.4 etc. \\
\hline
S2 & Permanent link to executables of this version  & For example: $https://github.com/combogenomics/$ $DuctApe/releases/tag/DuctApe-0.16.4$ \\
\hline
S3  & Permanent link to Reproducible Capsule & \\
\hline
S4 & Legal Software License & List one of the approved licenses \\
\hline
S5 & Computing platforms/Operating Systems & For example Android, BSD, iOS, Linux, OS X, Microsoft Windows, Unix-like , IBM z/OS, distributed/web based etc. \\
\hline
S6 & Installation requirements \& dependencies & \\
\hline
S7 & If available, link to user manual - if formally published include a reference to the publication in the reference list & For example: $http://mozart.github.io/documentation/$ \\
\hline
S8 & Support email for questions & \\
\hline
\end{tabular}
\caption{Software metadata (optional)}
\end{table}

\end{document}
\endinput
%%
%% End of file `SoftwareX_article_template.tex'.