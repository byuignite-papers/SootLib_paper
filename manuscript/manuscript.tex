%% 
%% Copyright 2007, 2008, 2009 Elsevier Ltd
%% 
%% This file is part of the 'Elsarticle Bundle'.
%% ---------------------------------------------
%% 
%% It may be distributed under the conditions of the LaTeX Project Public
%% License, either version 1.2 of this license or (at your option) any
%% later version.  The latest version of this license is in
%%    http://www.latex-project.org/lppl.txt
%% and version 1.2 or later is part of all distributions of LaTeX
%% version 1999/12/01 or later.
%% 
%% The list of all files belonging to the 'Elsarticle Bundle' is
%% given in the file `manifest.txt'.
%% 

%% Template article for Elsevier's document class `elsarticle'
%% with numbered style bibliographic references
%% SP 2008/03/01

\documentclass[preprint,12pt, a4paper]{elsarticle}

%% Use the option review to obtain double line spacing
%% \documentclass[authoryear,preprint,review,12pt]{elsarticle}

%% For including figures, graphicx.sty has been loaded in
%% elsarticle.cls. If you prefer to use the old commands
%% please give \usepackage{epsfig}

%% The lineno packages adds line numbers. Start line numbering with
%% \begin{linenumbers}, end it with \end{linenumbers}. Or switch it on
%% for the whole article with \linenumbers.

\usepackage{amssymb}
\usepackage{lineno}
\usepackage{amsmath}
\usepackage{float}
\usepackage[version=4]{mhchem}
\usepackage{booktabs}
\usepackage{siunitx}
\usepackage{bm}
\restylefloat{table}
\usepackage{hyperref}

\newcounter{bla}
\newenvironment{refnummer}{%
\list{[\arabic{bla}]}%
{\usecounter{bla}%
 \setlength{\itemindent}{0pt}%
 \setlength{\topsep}{0pt}%
 \setlength{\itemsep}{0pt}%
 \setlength{\labelsep}{2pt}%
 \setlength{\listparindent}{0pt}%
 \settowidth{\labelwidth}{[9]}%
 \setlength{\leftmargin}{\labelwidth}%
 \addtolength{\leftmargin}{\labelsep}%
 \setlength{\rightmargin}{0pt}}}
 {\endlist}

\journal{Computer Physics Communications}

\begin{document}

\begin{frontmatter}

%% Title, authors and addresses

%% use the tnoteref command within \title for footnotes;
%% use the tnotetext command for theassociated footnote;
%% use the fnref command within \author or \address for footnotes;
%% use the fntext command for theassociated footnote;
%% use the corref command within \author for corresponding author footnotes;
%% use the cortext command for theassociated footnote;
%% use the ead command for the email address,
%% and the form \ead[url] for the home page:
%% \title{Title\tnoteref{label1}}
%% \tnotetext[label1]{}
%% \author{Name\corref{cor1}\fnref{label2}}
%% \ead{email address}
%% \ead[url]{home page}
%% \fntext[label2]{}
%% \cortext[cor1]{}
%% \address{Address\fnref{label3}}
%% \fntext[label3]{}

\title{SootLib: a soot model library for combustion CFD}

%% use optional labels to link authors explicitly to addresses:
%% \author[label1,label2]{}
%% \address[label1]{}
%% \address[label2]{}

%\renewcommand{\thefootnote}{\fnsymbol{footnote}}
\author{Victoria B. Stephens}
\author{Josh Bedwell}
\author{David O. Lignell\corref{cor1}}

\cortext[cor1]{Corresponding author. \ead{davidlignell@byu.edu}}

\address{Department of Chemical Engineering, Brigham Young University, Provo, UT 84602, United States}

\begin{abstract}
SootLib abstract
%RadLib is a modular C++ library of radiation property models that can be applied to variety of systems involving radiative heat transfer, including CFD simulations. RadLib includes three major radiation property models---Planck Mean (PM) absorption coefficients, the weighted sum of gray gases (WSGG) model, and the rank-correlation spectral line weighted-sum-of-gray-gases (RCSLW) model---but its modular design and C++ and Python interface options permit convenient expansion to additional models. Several example cases illustrate the use of the models with an included ray-tracing solver and compare them in terms of accuracy relative to line-by-line solutions. The computational cost of the models is also compared. RadLib provides researchers with convenient access to validated radiation property models and a framework for further development.  

\end{abstract}

\begin{keyword}
%% keywords here, in the form: keyword \sep keyword
soot \sep combustion \sep reacting flows \sep CFD
\end{keyword}

\end{frontmatter}

\linenumbers

% PROGRAM SUMMARY.

{\bf PROGRAM SUMMARY}
  %Delete as appropriate.

\begin{small}
\noindent
{\em Program Title:} SootLib                                         \\
{\em CPC Library link to program files:} (to be added by Technical Editor) \\
{\em Developer's repository link:} https://github.com/BYUignite/sootlib \\
{\em Code Ocean capsule:} TODO \\
{\em Licensing provisions(please choose one):} MIT \\
{\em Programming language:} C++, Python                                   \\
{\em Nature of problem(approx. 50-250 words):} \\
TODO \\
{\em Solution method(approx. 50-250 words):}\\
TODO \\ 
{\em Additional comments including restrictions and unusual features (approx. 50-250 words):}\\
The library is intended to be used in Linux-like terminal applications.  
   \\
\end{small}
%% main text

%%%%%%%%%%%%%%%%%%%%%%%%%%%%%%%%%%%%%%%%%%%%%%%%%%%%%%%%%%%%%%%%%%%%%%%%%%%

\section{Introduction}
\label{s:intro}

text here \citep{Ashurst_2005}
%\begin{itemize}
%	\item Introduce the scientific background and the motivation for developing the software. \cite{Lignell_2018}
%	\item Explain why the software is important, and describe the exact (scientific) problem(s) it solves.
%	\item Indicate in what way the software has contributed (or how it will contribute in the future) to the process of scientific discovery; if available, this is to be supported by citing a research paper using the software.
%	\item Provide a description of the experimental setting (how does the user use the software?).
%	\item Introduce related work in literature (cite or list algorithms used, other software etc.).
%\end{itemize}

%%%%%%%%%%%%%%%%%%%%%%%%%%%%%%%%%%%%%%%%%%%%%%%%%%%%%%%%%%%%%%%%%%%%%%%%%%%

\section{Model Descriptions}
\label{s:models}

%--------------------------------------------------------------------------
\subsection{Chemistry}
\label{ss:chemistry}

%--------------------------------------------------------------------------
\subsection{Particle dynamics}
\label{ss:particle_dynamics}

%%%%%%%%%%%%%%%%%%%%%%%%%%%%%%%%%%%%%%%%%%%%%%%%%%%%%%%%%%%%%%%%%%%%%%%%%%%

\section{Software description}
\label{s:architecture}

%Describe the software in as much as is necessary to establish a vocabulary needed to explain its impact. 
%
%Give a short overview of the overall software architecture; provide a pictorial component overview or similar (if possible). If necessary provide implementation details.
%
%Present the major functionalities of the software.

%%%%%%%%%%%%%%%%%%%%%%%%%%%%%%%%%%%%%%%%%%%%%%%%%%%%%%%%%%%%%%%%%%%%%%%%%%%

\section{Validation and Examples}
\label{s:examples}

%Provide at least one illustrative example to demonstrate the major functions.

%%%%%%%%%%%%%%%%%%%%%%%%%%%%%%%%%%%%%%%%%%%%%%%%%%%%%%%%%%%%%%%%%%%%%%%%%%%

\section{Discussion}
\label{s:discussion}

%\textbf{This is the main section of the article and the reviewers weight the description here appropriately}
%
%Indicate in what way new research questions can be pursued as a result of the software (if any).
%
%Indicate in what way, and to what extent, the pursuit of existing research questions is improved (if so).
%
%Indicate in what way the software has changed the daily practice of its users (if so).
%
%Indicate how widespread the use of the software is within and outside the intended user group.
%
%Indicate in what way the software is used in commercial settings and/or how it led to the creation of spin-off companies (if so).

%%%%%%%%%%%%%%%%%%%%%%%%%%%%%%%%%%%%%%%%%%%%%%%%%%%%%%%%%%%%%%%%%%%%%%%%%%%

\section{Conclusions}
\label{s:conclusions}

Set out the conclusion of this original software publication.

%%%%%%%%%%%%%%%%%%%%%%%%%%%%%%%%%%%%%%%%%%%%%%%%%%%%%%%%%%%%%%%%%%%%%%%%%%%

\section{Conflict of Interest}
%Please select the appropriate text:
%
%Potential conflict of interest exists:
%We wish to draw the attention of the Editor to the following facts, which may be considered as potential conflicts of interest, and to significant financial contributions to this work. The nature of potential conflict of interest is described below: [Describe conflict of interest]

%No conflict of interest exists:
The authors declare that they have no known competing financial interests or personal relationships that could have appeared to influence the work reported in this paper.

%%%%%%%%%%%%%%%%%%%%%%%%%%%%%%%%%%%%%%%%%%%%%%%%%%%%%%%%%%%%%%%%%%%%%%%%%%%

\section*{Acknowledgements}
\label{}

%The authors extend special thanks to Hadi Bordbar for assistance with the WSGG model and to Vladimir Solovjov and Brent Webb for their insights and assistance with the RCSLW model. 
This research did not receive any specific grant from funding agencies in the public, commercial, or
not-for-profit sectors.

%%%%%%%%%%%%%%%%%%%%%%%%%%%%%%%%%%%%%%%%%%%%%%%%%%%%%%%%%%%%%%%%%%%%%%%%%%%

%% References:

\bibliographystyle{elsarticle-num} 
\bibliography{references} 

\end{document}
\endinput
